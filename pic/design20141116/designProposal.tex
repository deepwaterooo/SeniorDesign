% Created 2014-11-15 Sat 11:10
\documentclass[9pt,b5paper]{article}
\usepackage{graphicx}
\usepackage{xcolor}
\usepackage{xeCJK}
\usepackage{longtable}
\usepackage{float}
\usepackage{textcomp}
\usepackage{geometry}
\geometry{left=0cm,right=0cm,top=0cm,bottom=0cm}
\usepackage{multirow}
\usepackage{multicol}
\usepackage{listings}
\usepackage{algorithm}
\usepackage{algorithmic}
\usepackage{latexsym}
\usepackage{natbib}
\usepackage{fancyhdr}
\usepackage[xetex,colorlinks=true,CJKbookmarks=true,linkcolor=blue,urlcolor=blue,menucolor=blue]{hyperref}


\lstset{language=c++,numbers=left,numberstyle=\tiny,basicstyle=\ttfamily\small,tabsize=4,frame=none,escapeinside=``,extendedchars=false,keywordstyle=\color{blue!70},commentstyle=\color{red!55!green!55!blue!55!},rulesepcolor=\color{red!20!green!20!blue!20!}}
\author{Jenny Huang}
\date{\today}
\title{CS480 Senior Design Proposal}
\hypersetup{
  pdfkeywords={},
  pdfsubject={},
  pdfcreator={Emacs 24.3.1 (Org mode 8.2.7c)}}
\begin{document}

\maketitle
\tableofcontents

Meeting time: 11/15/2014 Saturday 11:00

\section{Goal:}
\label{sec-1}
Try to organize the outline, so that we can make sure we have a feasible/practical outline/plan for later on implement module by module.

\section{\textbf{QPushButton} instead of original class design \textbf{Pixel}}
\label{sec-2}
functions that we are going to use: 
\begin{itemize}
\item setCheckable(true);
\item setAutoExclusive(false);
\end{itemize}

\section{Keyboard key sets}
\label{sec-3}
\subsection{keyset define}
\label{sec-3-1}
\begin{center}
\begin{tabular}{ll}
\hline
Keys & Functions\\
\hline
Ctrl & select multiple buttons\\
Shift & select buttons in the middle\\
Ctrl + c & copy frames selcted\\
Ctrl + v & paste frames selected\\
\hline
\end{tabular}
\end{center}

\subsection{reading keyset reference}
\label{sec-3-2}
\begin{itemize}
\item \url{http://qt-project.org/doc/qt-4.8/qml-keyevent.html}
\item \url{http://qt-project.org/doc/qt-4.8/qkeysequence.html}
\begin{verbatim}
QKeyEvent* ke;
QString modifier = QString::null;
if (ke->modifiers() & Qt::ShiftModifier)
    modifier += "Shift+";
if (ke->modifiers() & Qt::ControlModifier)
    modifier += "Ctrl+";
if (ke->modifiers() & Qt::AltModifier)
    modifier += "Alt+";
if (ke->modifiers() & Qt::MetaModifier)
    modifier += "Meta+";
QString key = (QString)QKeySequence(ke->key());
QKeySequence result(modifier + key);
\end{verbatim}
\item \url{http://qtdocs.narod.ru/4.1.0/doc/html/qkeysequence.html}
\item \url{http://stackoverflow.com/questions/12830788/handle-key-events-ctrltab-and-ctrlshifttab}
\begin{verbatim}
Item {
    width: 100
    height: 100
    focus: true
    Keys.onPressed: {
        if(event.modifiers && Qt.ControlModifier) {
            if(event.key === Qt.Key_Tab) {
                console.log('forward')
                event.accepted = true;
            } else if(event.key === Qt.Key_Backtab) {            
                console.log('backward')
                event.accepted = true;
            }
        }
    }
}
\end{verbatim}

\item \url{http://stackoverflow.com/questions/17204142/capturing-modifier-keys-qt}
\begin{verbatim}
void MainWindow::wheelEvent( QWheelEvent *wheelEvent )
{
    if( wheelEvent->modifiers() & Qt::ShiftModifier )
    {
        // do something awesome
    }
    else if( wheelEvent->modifiers() & Qt::ControlModifier )
    {
        // do something even awesomer!
    }
}
\end{verbatim}
\item \url{http://www.qtcentre.org/archive/index.php/t-28754.html}
\item python \url{http://forums.opensuse.org/showthread.php/436964-How-to-catch-Shift-Ctrl-key-in-keyPressEvent}
\item \url{http://linux.die.net/man/3/qkeysequence}
\item \url{http://www.codeproject.com/Articles/7305/Keyboard-Events-Simulation-using-keybd-event-funct}
\item \url{http://forums.codeguru.com/showthread.php?447587-Detecting-if-Control-or-Shift-key-was-down-when-key-pressed}
\item win32 \url{http://www.codeproject.com/Articles/6819/SendKeys-in-C}
\item \url{http://alleg.sourceforge.net/stabledocs/en/alleg006.html}
\end{itemize}

\section{myGridLayout for left-handside grid and preview}
\label{sec-4}

\section{repository update history record}
\label{sec-5}
volunteer myself to do the record;  
% Emacs 24.3.1 (Org mode 8.2.7c)
\end{document}