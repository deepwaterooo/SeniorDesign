% Created 2014-12-04 Thu 18:30
\documentclass[9pt,b5paper]{article}
\usepackage{graphicx}
\usepackage{xcolor}
\usepackage{xeCJK}
\usepackage{longtable}
\usepackage{float}
\usepackage{textcomp}
\usepackage{geometry}
\geometry{left=0cm,right=0cm,top=0cm,bottom=0cm}
\usepackage{multirow}
\usepackage{multicol}
\usepackage{listings}
\usepackage{algorithm}
\usepackage{algorithmic}
\usepackage{latexsym}
\usepackage{natbib}
\usepackage{fancyhdr}
\usepackage[xetex,colorlinks=true,CJKbookmarks=true,linkcolor=blue,urlcolor=blue,menucolor=blue]{hyperref}


\lstset{language=c++,numbers=left,numberstyle=\tiny,basicstyle=\ttfamily\small,tabsize=4,frame=none,escapeinside=``,extendedchars=false,keywordstyle=\color{blue!70},commentstyle=\color{red!55!green!55!blue!55!},rulesepcolor=\color{red!20!green!20!blue!20!}}
\author{Jenny Huang}
\date{\today}
\title{CS480 Tower iLLuminati Design Report}
\hypersetup{
  pdfkeywords={},
  pdfsubject={},
  pdfcreator={Emacs 24.3.1 (Org mode 8.2.7c)}}
\begin{document}

\maketitle
\tableofcontents

cover sheet -  page 1
\begin{center}
\begin{tabular}{ll}
\hline
Project Title & Tower iLLuminati\\
Sponsor & Dr. Robert Rinker\\
Team name & Heyan Huang\\
\hline
\end{tabular}
\end{center}
\newpage

\begin{abstract}
Execute Summary 1/2 page

Our current software is the v0.3 version writen in C\#. And every "window" change needs a click. And the v0.2 was black and white, and since UI CS Curriculum is C++ based, we want to try our best to make a most basic software so that later on every ACM member with interest would be able to add any components that they are interested in and so that there won't be programming language barrier for them. 

It's a old project that repeats each year maybe just for us students to get practised. Our current version of project was pretty much a fully functioning one, and it was writen by one ACM student who graduated already, and it was writen in C\# with his own interest.
\end{abstract}

\section{Project Background}
\label{sec-1}
\begin{itemize}
\item Motivation for the work

Our cliend Dr. Rinker is very environment-friendly on campus, and he lead the ACM team for the department, and he enjoys offering music-related fun activities for the compus like tower light show for homecoming events. For tower light show, the current tower animator c\#-programming based software plays the .tan light file type separately from the .wav audio file, which is not convenient for him and his ACM parties to use.

\item Identify the need

Since University of Idaho Computer Science department is mainly c++-based programming, and the c\# language does produce language barrier for a certain amount of ACM users, and the software is quite some distance from user-friendly and functionaly complete, there exist the need for potential refactor, reimplementaion, or update on this software.

\item Describe the expected benefits

Reimplement and update the software in c++ language will promote ACM user attendence on product software updates and also, we, as the senior design team would be able to get good practise and benefit a long way from various aspects like programming technical practise, softare engineering, project design and handling, as well as problem-solving skills. 
\begin{figure}[htb]
\centering
\includegraphics[width=.9\linewidth]{../pic/oldVersion.png}
\caption{Base software Interface}
\end{figure}
\end{itemize}

\section{Problem Definition - 1 page}
\label{sec-2}
\subsection{Goals and deliverables}
\label{sec-2-1}
\begin{itemize}
\item Link .tan file with corresponding audio file
\item Rewrite in C++
\item Make light shows easier
\begin{itemize}
\item Less clicks
\item Automated movement
\item Pre-generated shapes
\item Pre-generated animations
\item import from other files
\end{itemize}
\end{itemize}
\subsection{Specifications \& constraints}
\label{sec-2-2}


\section{Project Plan  - 1/2 page}
\label{sec-3}
\subsection{Tasks and schedule}
\label{sec-3-1}
\subsection{Team responsibilities}
\label{sec-3-2}

\section{Concepts considered - 3 pages}
\label{sec-4}
\subsection{Original ideas + those derived from other sources}
\label{sec-4-1}
\begin{table}[htb]
\caption{System Functions conpressed into Menubar}
\centering
\begin{tabular}{llllll}
\hline
File & Edit & Play & Select & Mtool & Help\\
\hline
Open & Copy (C-c) & play from start & Row (SPC-mse) & Insert Shapes & Documentation\\
New & Cut (C-x) & play from current & Col (Sft-SPC-mse) & Define Pattern* & About\\
Save & Paste (C-v) & Pause & All (C-a) & Color Gradient* & \\
Save as & Insert After/Before & Stop & Invert Slt  (C-i) &  & \\
Export & Clear & Move Forward & Slt Shift-C &  & \\
Close & Delete & Move Backward & Slt C-mouse &  & \\
Exit & Undo (C-z) &  &  &  & \\
 & Redo (C-y) &  &  &  & \\
\hline
\end{tabular}
\end{table}
\subsection{Quantitative data or measurements}
\label{sec-4-2}
References: Literature/Catalog Review


\section{concept Selection - 1 page}
\label{sec-5}
\subsection{GUI Layout Selection}
\label{sec-5-1}
We have been proposed two GUI designs, snapshotted when we discussed about them, which are listed as followed. 
\begin{figure}[htb]
\centering
\includegraphics[width=.9\linewidth]{../pic/Screenshot_from_2014-12-02_22:43:29_one.png}
\caption{team manager proposed new design}
\end{figure}
\begin{figure}[htb]
\centering
\includegraphics[width=.9\linewidth]{../pic/Screenshot_from_2014-12-02_23:04:43_me.png}
\caption{Original traditional design}
\end{figure}

\subsection{How did you arrive at your final selection?}
\label{sec-5-2}
\subsection{Include morphological charts, decision matrices}
\label{sec-5-3}

\section{System Architecture - 2 pages}
\label{sec-6}
\subsection{Describe the conceptual design ' justify continued development}
\label{sec-6-1}
\subsection{Describe the components and how they are integrated}
\label{sec-6-2}
\subsubsection{Highlight novel features ' your value added}
\label{sec-6-2-1}
\subsubsection{How does each major component satisfy a requirement}
\label{sec-6-2-2}
\subsection{Provide quantitative results from tests or analysis}
\label{sec-6-3}
\subsection{Architecture Design}
\label{sec-6-4}
The software can be described
\section{Human Interface Design}
\label{sec-7}
\begin{figure}[htb]
\centering
\includegraphics[width=.9\linewidth]{../pic/Screenshot_from_2014-12-02_19:22:35.png}
\caption{Main Windows}
\end{figure}
\begin{figure}[htb]
\centering
\includegraphics[width=.9\linewidth]{../pic/Screenshot_from_2014-11-27_23:34:06.png}
\caption{Pop-up Windows}
\end{figure}

\section{Future Work - 1/2 page}
\label{sec-8}
\subsection{Recommendations  for continued work}
\label{sec-8-1}
\subsubsection{Features that didn't find their way into the current design}
\label{sec-8-1-1}
\subsubsection{Estimate size and duration of the required effort}
\label{sec-8-1-2}
\begin{itemize}
\item Implement functionality to GUI
\item Implement multiple pixel movement
\item Figure out modifications to .tan file layout
\item Next Semester
\begin{itemize}
\item Implement more advanced animations/movements
\item Add more GUI features
\end{itemize}
unreolved issues and plan for attacking these
\end{itemize}

\section{Appendices}
\label{sec-9}
The supporting documents to long or detailed for main body includes the following several sections. 
\subsection{Calculations \& drawings}
\label{sec-9-1}
\subsection{Large tables \& figures}
\label{sec-9-2}
\begin{figure}[htb]
\centering
\includegraphics[width=.9\linewidth]{../docs/Class_img03.png}
\caption{System Architecture Class Diagram}
\end{figure}
\begin{figure}[htb]
\centering
\includegraphics[width=.9\linewidth]{../docs/state.png}
\caption{Software State Diagram}
\end{figure}
\begin{table}[htb]
\caption{Commands and Fast keys}
\centering
\begin{tabular}{lll}
\hline
MainMenu & Commands & Fast Keyset\\
\hline
File & File & C-f\\
 & New & \\
 & Open & C-o\\
 & Save & \\
 & Save as & \\
 & Export & \\
 & Close & \\
 & Exit & \\
Edit & Edit & C-e\\
 & Cut & C-x\\
 & Copy & C-c\\
 & Paste & C-v\\
 & Insert After/Before & \\
 & Delete & \\
 & Clear Frame & \\
 & Undo & C-z\\
 & Redo & C-y\\
Play & Play & C-p\\
 & From Beginning & C-b\\
 & From Current & C-n (now)\\
 & Pause & \\
 & Stop & \\
 & Move Forward & \\
 & Move Backward & \\
 & Preview Mode & C-r (review)\\
Select & Select & C-s\\
 & Row & SPC-mse\\
 & Col & Sft-SPC-mse\\
 & All & C-a\\
 & Invert & C-i\\
 & Deselect & C-d\\
Mtool & Mtool & C-m (movie)\\
 & Insert Shapes & \\
 & Define Pattern* & \\
 & Color Gradient* & \\
Help & Help & C-h\\
 & Documentation & \\
 & About & \\
\hline
\end{tabular}
\end{table}

Visualization (sketches, drawings, diagrams)
\begin{itemize}
\item Classes and Prototypes
Modeling and/or Experimentation
\end{itemize}
\subsection{Vendor data sheets}
\label{sec-9-3}
\subsection{Computer Programs}
\label{sec-9-4}
\begin{itemize}
\item Coding
\begin{itemize}
\item Based on Design
\item Testing
\end{itemize}
\end{itemize}
% Emacs 24.3.1 (Org mode 8.2.7c)
\end{document}